\UseRawInputEncoding
\documentclass[reprint, amsmath, amssymb, aps]{revtex4-2}

\usepackage{graphicx}% Include figure files
\usepackage{dcolumn}% Align table columns on decimal point
\usepackage{bm}% bold math
\usepackage{hyperref}% add hypertext capabilities
\usepackage{bibspacing}
\usepackage[font=scriptsize,labelfont=bf, justification=justified]{caption}% change fontsize in captions
\usepackage{float}
\usepackage[english]{babel}
\usepackage{booktabs}% cool table style
\hypersetup{
	colorlinks=true,       % false: boxed links; true: colored links
	linkcolor=black,        % color of internal links
	citecolor=black,        % color of links to bibliography
	filecolor=black,     % color of file links
	urlcolor=black         
}

\begin{document}
	
\title{PHYC30170 Physics with Astronomy and Space Science Lab 1;\\The Brusselator - A Computational Example of Chemical Oscillations}

\author{Daragh Hollman}
\email{daragh.hollman@ucdconnect.ie}
\date{\today}

\begin{abstract}
This is the abstract...
\end{abstract}

\maketitle

\section{Introduction}
A chemical oscillator is a non-linear system of reacting chemicals in which exhibits oscillations in the concentrations of their chemicals \cite{nicolis}. Non-linear systems have many applications in modern areas of science and engineering \cite{parada} and find use in non-linear control systems such as periodic drug delivery \citep{oscillationsAdvances}. Due to their complex nature, these kinds of systems can be difficult to set up and maintain as they tend to be very sensitive any changes in their environment conditions. This sensitivity does however does provide some use in the field of analytical chemistry as trace chemicals might upset the oscillatory process \cite{analytical}. The Brusselator is one such system of chemical oscillations and is the focus of this report.

\subsection{Chemical Equations}
Equations describing chemical reactions can be written as follows:
\begin{equation}
	A + B \rightarrow C + D
	\label{eq:chemicalEq}
\end{equation}where chemicals $A$ and $B$ interact with each other to form chemicals $C$ and $D$. This reaction will occur at a specific rate defined by an expression known as a rate equation \cite{dynamics}. It is common to assume that any chemical system will follow mass action kinetics \citep{dynamics}\cite{massAction}, that the reaction rate is direction proportional to the concentration of the reactants. A rate equation for the system in equation \ref{eq:chemicalEq} would look similar to:
\begin{equation}
	-\frac{dA}{dt} = k[A][B]
\end{equation}where $[A]$ and $[B]$ are the concentrations of chemicals A and B respectively and $k$ is the rate constant of the reaction. A negative sign is used to denote that A is decreasing in concentration. In this report we will consider the rate constant to be unity for simplicity.

\subsection{The Brusselator System}
The chemical equations of the Brusselator are typically described as follows \cite{manual}:
\begin{align}
	\begin{aligned}
	A &\rightarrow X & (a)\\
	B + X &\rightarrow Y + D & (b)\\
	2X + Y &\rightarrow 3X & (c)\\
	X &\rightarrow C & (d)
	\end{aligned}
\end{align}

With ODEs given by:
\begin{align}
	\begin{aligned}
	\frac{dX}{dt} &= A - (B + 1)X + X^2 Y & (a)\\
	\frac{dY}{dt} &= BX - X^2 Y & (b)
	\end{aligned}
	\label{eq:rate}
\end{align}

The steady state solution of this system is one which stays stationary over time, sometimes referred to as a stable point. At any stable point, the rate of change of $X$ and $Y$ is zero.
\begin{equation}
	\frac{dX}{dt}=0\,\text{  ;  }\frac{dY}{dt}=0
\end{equation}Hence we can find the stable point by solving for $X$ and $Y$. A full derivation is included in appendix 1, however a single point at $(X, Y) = \left(A, \frac{B}{A}\right)$ was calculated to be the only stable point in the system.\\

In this report we will investigate the evolution of the Brusselator system over time, and discuss the oscillatory nature of the reaction using phase space diagrams and concentration diagrams. This will be carried out over a range of initial conditions for X and Y, but also varying ratios of A and B.

\section{Computational Methods}

\subsection{The Euler Method}
The Euler method was chosen to numerically integrate the rate equations to evolve the system over time. The Euler method is used to solve the first-order initial value problem \cite{eulerError}:
\begin{equation}
	\frac{dy}{dx} = f\left(x, y \right),\, y(x_0) = y_0
\end{equation}Here we have a first-order ordinary differential equation with a known initial condition. Euler's method makes use of a relatively simple process which takes the slope of the function at an initial point and assumes a linear path between that point and the next some arbitrary step away. The formula is given as follows \cite{paulsNotes}:
\begin{equation}
	y(x+h) = y(x) + h f(x, y)
	\label{eq:eulers}
\end{equation}where $h$ is the step size. This will construct the tangent at coordinate $x$, and find the value of $y(x+h)$ to determine the next point. Hence to use Euler's method we can pick a starting point around which we want to approximate and then evaluate equation \ref{eq:eulers} until we have reached the desired number of steps.

\subsubsection{The Application of the Euler method to the System}

The Brusselator system is has two dependent variables which vary with time as shown in the rate equations, see equation \ref{eq:rate}, and hence we need to run two calculations of Euler's method simultaneously. Rewriting these rate equations in the form of equation \ref{eq:eulers}, we have the following:
\begin{align}
	\begin{aligned}
	X_{i+1} &= X_i + \Delta t \frac{dX}{dt} & (a)\\
	Y_{i+1} &= Y_i + \Delta t \frac{dY}{dt} & (b)
	\end{aligned}
	\label{eq:application}
\end{align}where $X_{i}$ is value of $X(t)$ and $X_{i+1}$ is the next step, $X(t+\Delta t)$ with step size $\Delta t$. These variables have the same meaning for equation \ref{eq:application}b but in terms of $Y$.

\subsection{Error Analysis of the Euler Method}
\subsubsection{Round-off Error}
In this simulation two types of error are introduced in the calculations, round-off error and truncation error. Although the round-off error is negligible compared to the truncation error of this simulation, it is important to reference in the context of a computational report. By default Python uses 64 bits to represent a floating point number \cite{python}. One for the sign, 11 for the exponent and 52 for the fraction. This means that we can only accurately represent a maximum of $2^{1024}$ and a minimum of $2^{-1024}$ \cite{niels}. This range is entirely sufficient for the purposes of this report and hence any errors due to round-off are negligible.

\subsubsection{Truncation Error}
Aside from the round-off error there exists a truncation error between the exact solution and the solution estimated by Euler's method. There is the local error between each step and the global error which is the summation of all the local errors up to a certain step \cite{owkes}. If we take the Taylor series approximation for a function:
\begin{equation}
	y_{i+1} = y_{i} + y'_{i}h + \frac{y''_i}{2} h^2 + \dots + \frac{y_i^{(n)}}{n!} h^n + \dots
\end{equation}we can clearly see that the first two terms of this are the same as Euler's method shown in equation \ref{eq:eulers}. The Taylor series will be an exact solution if all terms are included however if we truncate it after the first two terms, what remains will be the difference between Euler's method and the exact solution over a step. This is the local error, and for an ODE with:
\begin{align*}
	\begin{aligned}
	\frac{dy}{dx} &= f(x,y)\\
	y' &= f(x,y)
	\end{aligned}
\end{align*}we have:
\begin{equation}
	y_{i+1} = \underbrace{y_i + f(x_i, y_i) h}_\text{Euler's Method} + \underbrace{\frac{f'(x_i,y_i)}{2}h^2 + \dots}_\text{Error}
\end{equation}We can assume that, with a small step size $h$, that higher order terms will be small. Hence we can take the local error to scale with $\mathcal{O}(h^2)$ and that terms of $\mathcal{O}(h^3)$ or higher are negligible.\\

The global error is the summation of the local errors for each step. As the step size decreases, the number of steps within a length L increases with $h^{-1}$. Hence the global error is given by the following:
\begin{equation}
	\epsilon_g = \frac{1}{h} \sum \epsilon_l
\end{equation}and as the local error scales with $h^2$ and the global error scales with the local error and $h^{-1}$. The global error on the system for any number of steps scales with $\mathcal{O}(h)$, the step size.

\section{Results and Discussion}

\subsection{Varying the initial conditions}
What were the step and variable initial conditions.


\begin{figure*}
\includegraphics[width=1.8\columnwidth]{combinedPlot.png}
\caption{\label{fig:combinedPlot}Example caption}
\end{figure*}

\begin{figure*}
\includegraphics[width=1.8\columnwidth]{combinedPlot_fallToStable.png}
\caption{\label{fig:combinedPlot}Falls to stable point}
\end{figure*}

\begin{figure}
\includegraphics[width=0.85\columnwidth]{variationOfInitialConditions_phase.png}
\includegraphics[width=0.85\columnwidth]{variationOfInitialConditions_evolution.png}
\caption{\label{fig:combinedPlot}Variation of initial conditions}
\end{figure}

\section{Conclusion}

\clearpage
\bibliography{chemOscillationsReferences.bib}

\clearpage

\section*{Appendix 1 - Derrivation of the Stable Point}
\begin{align*}
	\begin{aligned}
	A - (B + 1)X + X^2 Y &= 0\\
	BX - X^2 Y &= 0\\
	\\
	\therefore \hspace{1cm} X^2 Y &= BX\\
	\\
	\implies A - (B + 1)X + BX &= 0\\
	X &= A\\
	\\
	A^2 Y &= BA\\
	\implies Y &= \frac{B}{A}\\
	\\
	(X, Y) &= \left(A, \frac{B}{A}\right)
	\end{aligned}
\end{align*}

\end{document}



