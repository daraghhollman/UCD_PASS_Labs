\documentclass[a4paper,12pt,twocolumn]{article}

% Packages
\usepackage{graphicx}

% Making cites superscript
\usepackage[sorting=none]{biblatex}
\DeclareCiteCommand{\supercite}[\mkbibsuperscript]
{\iffieldundef{prenote}
	{}
	{\BibliographyWarning{Ignoring prenote argument}}%
	\iffieldundef{postnote}
	{}
	{\BibliographyWarning{Ignoring postnote argument}}}
{\usebibmacro{citeindex}%
	\bibopenbracket\usebibmacro{cite}\bibclosebracket}
{\supercitedelim}
{}
\let\cite=\supercite

% Adding the bibliography file
\addbibresource{references.bib}

\begin{document}
	
	\begin{titlepage}
		\begin{center}
			
			\thispagestyle{empty}
			
			\Huge{
				\textbf{UCD School Of Physics}
			}
			
			\vspace{1cm}	
			
			\includegraphics[scale=0.08]{UCDLogo.png}
			
			\vspace{1cm}
			
			\large{
				\textbf{PHYC30170 Physics with Astronomy and Space Science Lab 1; \\
					CCDs and Spectroscopy \\
					\vspace{1cm}
					18/10/2022 \\
					\vspace{1cm}
					Daragh Hollman}
			} \\
			
		\end{center}
	\end{titlepage}
	
	\twocolumn[
	\begin{@twocolumnfalse}
		\begin{abstract}
			The aim of this experiment was to calibrate a CCD for spectroscopy and determine the resolution of a spectrograph. This was done by comparing the emission spectrum of a mercury arc lap to reference values... INSERT RESULTS.
		\end{abstract}
	\end{@twocolumnfalse}
	]
	
	\section{Introduction}
	
	\section{Theory}
		A diffraction grating is used to split incident light into its separate wavelengths. As a diffraction grating is an array of very narrow and evenly spaced slits, the diffraction pattern from each slit interferes such that the light disperses by a angle $\theta$ as described by equation \ref{eq:diffraction}\cite{hyperPhysics}.
		\begin{equation}
			n \lambda = d sin\theta
			\label{eq:diffraction}
		\end{equation} where $d$ is the spacing between the slits, $\lambda$ is the wavelength of the incident light, $\theta$ is the angle which the light is diffracted by and $n$ is a positive integer.
	
		Diffraction grating and equation + figure, what is the spectrograph setup, arc lamps + emission lines
	
	\section{Methodology}
		\subsection{Apparatus}
			Photo of experimental setup + focal lengths of all pieces, Atik 314L+ CCD
	
		\subsection{Determining the Readnoise and the Gain}
		
		\subsection{Wavelength Calibration}
	
		\subsection{Determining the Resolution of the Spectrograph}
	
	\section{Results and Analysis}
	
	\section{Conclusion}
	
	\printbibliography
\end{document}