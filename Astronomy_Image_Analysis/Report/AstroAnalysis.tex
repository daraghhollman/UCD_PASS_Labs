% Requires running Bibtex

\documentclass[%
reprint,
amsmath,amssymb,
aps,
]{revtex4-2}

\usepackage{graphicx}% Include figure files
\usepackage{dcolumn}% Align table columns on decimal point
\usepackage{bm}% bold math
\usepackage{hyperref}% add hypertext capabilities
\hypersetup{
	colorlinks=true,       % false: boxed links; true: colored links
	linkcolor=black,        % color of internal links
	citecolor=black,        % color of links to bibliography
	filecolor=black,     % color of file links
	urlcolor=black         
}

\begin{document}
	
	\preprint{APS/123-QED}
	
	\title{PHYC30170 Physics with Astronomy and Space Science Lab 1;\\Astronomical Image Analysis}% Force line breaks with \\
	
	\author{Daragh Hollman}
	\email{daragh.hollman@ucdconnect.ie}
	
	\date{\today}
	
	\begin{abstract}
		An article usually includes an abstract, a concise summary of the work
		covered at length in the main body of the article. 
	\end{abstract}

	\maketitle
	
	\section{Introduction}
	
		This is the introduction.\\
		
		IAC 80 Telescope at the Teide Observatory on Tenerife.
	
	
	\section{Theory}
	
		This is the theory section.
	
		\subsection{Image Reduction}
			\subsection{Bias Images}
			
			\subsection{Flat Field Images}
		
		\subsection{Aperture Photometry}
		
			\begin{equation}
				m_{std} = -2.5 \, \log_{10}\left(\frac{F}{t}\right) + \text{ZP}
				\label{eq:mag}
			\end{equation} where $m_{std}$ is the calibrated magnitude of a source in the system, $F$ is the measured background-subtracted counts from the light source, $t$ is the exposure time and $\text{ZP}$ is the zeropoint for the image.\\
		
			Do some talking about what the zeropoint actually is.\\
			
			Rearranging equation \ref{eq:mag} for $\text{ZP}$:
			
			\begin{equation}
				\text{ZP} = m_{std} + 2.5 \log_{10}\left(\frac{F}{t}\right)
				\label{eq:zeropoint}
			\end{equation}
		
		\subsection{CCD}
		
			A CCD is used...\\
			
			Signal to noise (which can be compared to the fits file data)...\\
			
			\begin{equation}
				\frac{S}{N} = \frac{F_\star}{\sqrt{F_\star + n_\star \left( 1 + \frac{n_\star}{n_\text{sky}}\right) \times \left(F_\text{sky} + R^2\right)}}
			\end{equation}where $\frac{S}{N}$ is the signal noise ratio, $F_\star$ and $F_\text{sky}$ are the number of counts measured from the source and the background annulus respectively, $n_\star$ and $n_\text{sky}$ are the number of pixels within the  
	 
	\section{Methodology}
	
		\subsection{Data Reduction}
		
		\subsection{Determining the Size of Messier 91}
			
			\subsubsection{Angular Size}
			
			\subsubsection{Adjusting for Inclination and Position Angle}
		
		
		\subsection{NN Serpentis}
		
		
	\section{Results}
	
		\subsection{Messier 91}
		
		\subsection{NN Serpentis}

	\section{Conclusion}
		
		
	\bibliography{apssamp}% Produces the bibliography via BibTeX.
		
	\appendix
		
	\section{Python Code}
		
		
		
\end{document}

