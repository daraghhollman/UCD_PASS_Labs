\documentclass[%
reprint,
amsmath,amssymb,
aps,
]{revtex4-2}

\begin{document}
	\title{PHYC30170 Physics with Astronomy and Space Science Lab 1;\\
	Ramsauer-Townsend; Statement of Intent}
	
	\author{Daragh Hollman}
	
	\date{\today}
	
	\maketitle
	
	\onecolumngrid
	\section{What is the aim of the experiment?}
		The aim is to demonstrate the Ramsauer-Townsend effect, determining the minimum scattering probability and ionisation point.
	
	\section{What measurements should be made and how?}
		The voltage across the thyratron will be varied from $0\,\text{V}$ to $15\,\text{V}$. The plate voltage and shield voltage will be measured. Liquid nitrogen will be used to reduce the pressure of the xenon in the apparatus and the measurements will be repeated.
	
	\section{How will the final result be obtained from the experimental data?}
		The plate and shield currents will be determined from their respective voltages and resistors using ohms law. The probability of scattering will be calculated from the ratio of the plate and shield currents.
	
	\section{What are the main safety concerns with the experiment and precautions that should be taken?}
		Working with electricity and liquid nitrogen have intrinsic risks and so precaution is needed. Care is needed when handling the electrical components and pouring and handling the liquid nitrogen.
	
\end{document}