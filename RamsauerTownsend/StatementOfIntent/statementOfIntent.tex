\documentclass[%
reprint,
amsmath,amssymb,
aps,
]{revtex4-2}

\begin{document}
	\title{PHYC30170 Physics with Astronomy and Space Science Lab 1;\\
	Ramsauer-Townsend; Statement of Intent}
	
	\author{Daragh Hollman}
	
	\date{\today}
	
	\maketitle
	
	\onecolumngrid
	\section{What is the aim of the experiment?}
		To demonstrate the Ramsauer-Townsend effect, showing that the minimum scattering probability and ionisation point agrees with the theory.
	
	\section{What measurements should be made and how?}
		Liquid nitrogen will be used to reduce the pressure of the xenon in the apparatus. The plate current will be measured  with varying the voltage source from $0 \,\text{V}$ to $15 \,\text{V}$.
	
	\section{How will the final result be obtained from the experimental data?}
		The electron energy can be calculated from the difference in the voltage and the shield voltage
	
	\section{What are the main safety concerns with the experiment and precautions that should be taken?}
		Precaution is needed when working with lasers and when working in the dark room. Do not look directly at the laser. Make sure the area surrounding the apparatus is clear of hazards such as bags and coats on the floor.
	
\end{document}